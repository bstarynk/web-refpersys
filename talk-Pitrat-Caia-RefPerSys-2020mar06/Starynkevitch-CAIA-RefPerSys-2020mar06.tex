%%%%%%%%%%%%%%%%%%%%%%%%%%%%%%%%%%%%%%%%%%%%%%%%%%%%%%%%%%%%%%%%%%%%%%%
% file Starynkevitch-CAIA-RefPerSys-2020mar06.tex
% github.com/bstarynk/web-refpersys
% subdirectory piotr ~/web-refpersys/
% inspired by http://cse.unl.edu/~cbourke/latex/UNLTheme.tex
%%%%%%%%%%%%%%%%%%%%%%%%%%%%%%%%%%%%%%%%%%%%%%%%%%%%%%%%%%%%%%%%%%%%%%%

\documentclass[xcolor=svgnames,final,smaller,a4]{beamer}
\usepackage{relsize}
\usepackage{luacode}
\usepackage{xcolor}
\usepackage{alltt}
\usepackage{wasysym}
\usepackage{hyperref}


\hypersetup{
  colorlinks   = true, %Colours links instead of ugly boxes
  urlcolor     = NavyBlue, %Colour for external hyperlinks
  linkcolor    = DarkGreen, %Colour of internal links
  citecolor   = DarkMagenta, %Colour of citations
  frenchlinks = true,
}

\usetheme{AnnArbor}

\title[From CAIA to RefPerSys]{\textsc{From CAIA to RefPerSys} \\
  reflexive, introspective, meta- based AI systems}

\author[B.Starynkevitch \ldots]{Basile \textsc{Starynkevitch} - \href{http://starynkevitch.net/Basile/}{\texttt{starynkevitch.net/Basile}}\\ \href{mailto:basile@starynkevitch.net}{\color{blue}{\texttt{basile@starynkevitch.net}}} - \textsc{92340  Bourg-La-Reine}, France}
  \institute{\textit{RefPerSys} - \href{http://refpersys.org}{\color{red}{\texttt{refpersys.org}}}}

  \date[March 6 \textsuperscript{th}, 2020]{Paris, March 6\textsuperscript{th}, 2020 - seminar in honor of the late J.Pitrat}

 \begin{luacode*}
   local gitpip=io.popen("git log --no-color --format=oneline -1 --abbrev=16 --abbrev-commit -q | cut -d' ' -f1")
   gitid=gitpip:read()
   gitpip:close()
 \end{luacode*}
 \newcommand{\mygitid}{\luadirect{tex.print(gitid)}}
 \newcommand{\RefPerSys}{\href{http://refpersys.org}{\textsc{RefPerSys}}}

 %%%%%%%%%%%%%%%%%%%%%%%%%%%%%%%%%%%%%%%%%%%%%%%%%%%%%%%%%%%%%%%%
  \begin{document}

 \begin{frame}
   \begin{relsize}{-0.5}
     \titlepage
     
     {\textcolor{brown}{{\large \textbf{Opinions are only mines}}}} \hspace{0.5cm}{\relsize{-1}{git commit \texttt{\mygitid}}}

   \begin{relsize}{-0.5}
       {these slides are under} \raisebox{-0.2cm}{\includegraphics[width=0.12\textwidth]{CC-BY-SA-4}} \href{https://creativecommons.org/licenses/by-sa/4.0/}{Creative Commons Attribution-ShareAlike 4.0 International}
        
        My employer {\relsize{-0.5}{(or funding agencies at work)}} would probably
        disagree with most of my opinions here. Any agreement with my employer's
        policies or positions is accidental.
        
   \end{relsize}
   
   \end{relsize}
 \end{frame}

\begin{frame}{Overview}
\tableofcontents
\end{frame}

 \section{What is AI?}
 \label{sec:what-is-ai}
 
 \begin{frame}
   \frametitle{What is AI?}
   \begin{itemize}
   \item \textcolor{red}{\large \textbf{A}}rtificial
     \textcolor{red}{\large \textbf{I}}ntelligence (and
     \href{https://en.wikipedia.org/wiki/AI_winter}{AI winter} - predicted by J.Pitrat)
     \begin{itemize}
       \item \textbf{A}rtificial \textbf{G}eneral \textbf{I}ntelligence 
       \item symbolic \textbf{A}rtificial \textbf{I}ntelligence
       \item machine learning  \textbf{A}rtificial \textbf{I}ntelligence
       \item\textbf{A}rtificial \textbf{I}ntelligence applications
     \end{itemize}
   \item \textcolor{red}{\large \textbf{A}}dvanced  \textcolor{red}{\large \textbf{I}}nformatics
   \item \textbf{A}bstract \textbf{I}nterpretation {\relsize{-1}{(a
       technique for static program analysis, by
       Cousot)}}\\
     $\rightarrow$ IMHO it could be the next ``AI winter''\\
     \relsize{-1}{I am impatiently waiting for
       \href{http://frama-c.com/}{Frama-C} fully automated analysis of
       \href{https://www.tensorflow.org/}{\textsc{TensorFlow}} C++
       code (its \href{https://floating-point-gui.de/}{floating point} precision issues), or of
       \href{https://github.com/bstarynk/caia-pitrat/}{\textsc{Caia}}
       C code {\large \smiley{}}; the ``\textcolor{brown}{robust AI}'' buzzword....}
   \end{itemize}


   \bigskip

   \begin{block}{philosophical question}
     Is AI a science unrelated to computer science, or is it part of it?
   \end{block}

   \begin{relsize}{-1.5}
     (probable major disagreement between J.Pitrat and me)
   \end{relsize}
   
 \end{frame}
     
 \begin{frame}
   \frametitle{Why AI systems are needed? hard problems (1/2)}

   Because mankind or nations or continents -and decision makers-
   face major problems which are not fully understood (incomplete list) :

   \begin{itemize}
   \item \href{https://en.wikipedia.org/wiki/Global\_warming}{global
     warming},
     \href{https://en.wikipedia.org/wiki/Malnutrition}{malnutrition}
     and \href{https://en.wikipedia.org/wiki/Pollution}{pollution}
   \item the [co-] design and operation of complex systems (nuclear
     power plants -\href{https://en.wikipedia.org/wiki/ITER}{ITER}-,
     autonomous transportation,
     \href{https://en.wikipedia.org/wiki/Smart_grid}{smart grids}, \href{https://en.wikipedia.org/wiki/Super_grid}{super grid},
     \href{https://en.wikipedia.org/wiki/Smart_city}{smart cities},
     communication networks,
     \href{https://en.wikipedia.org/wiki/Water\_distribution\_system}{water
       distribution system},
     \href{https://en.wikipedia.org/wiki/Cobot}{cobots})
   \item managing complex systems (e.g. the
     \href{https://en.wikipedia.org/wiki/World\_Wide\_Web}{World Wide
       Web}) or their implementation (fiber optics deployment in
     France) or organizations
     (\href{https://www.velib-metropole.fr/}{Vélib}) - avoiding \href{https://en.wikipedia.org/wiki/Boreout}{boreout}
     \item \href{https://en.wikipedia.org/wiki/Digital_twin}{digital
       twins} (e.g. of an automobile, of holiday travels) and
       distributed
       \href{https://en.wikipedia.org/wiki/Embedded_system}{embedded
         systems} (or
       \href{https://en.wikipedia.org/wiki/Edge_computing}{edge
         computing})
     \item macro-economical policies of the Euro zone (why negative interest rates?)
     \item fighting global bio-viruses (e.g. coronavirus = closed
       software + chemistry)
   \item in France: retirement policies (why is it difficult to simulate?)
   \end{itemize}

 \end{frame}

 
 \begin{frame}
   \frametitle{Why AI systems are needed? hard problems (2/2)}

   \begin{itemize}
   \item in general: defining useful regulations (e.g. $CO_2$ global
     market, world-wide distribution of economical wealth,
     \href{https://en.wikipedia.org/wiki/Neuroeconomics}{neuroeconomics})
   \item political problems: peace in Middle East?
   \item lack of motivated software developers (how to keep their
     motivations and productivity?) - see \href{https://en.wikipedia.org/wiki/Bullshit_Jobs}{\textit{Bullshit jobs}} and \href{https://www.editions-observatoire.com/content/La_comédie_inhumaine}{\textit{La comédie (in)humaine}}
   \item going to Mars, \href{https://en.wikipedia.org/wiki/Terraforming\_of\_Mars}{terraforming it}, \href{https://en.wikipedia.org/wiki/Colonization_of_Mars}{colonizing it}
   \item understanding the physical world (particle physics, quantum
     physics, exobiology, cosmology?) and improving it
     (\href{https://en.wikipedia.org/wiki/Climate-smart_agriculture}{climate-smart
       agriculture}, world-wide better
     \href{https://en.wikipedia.org/wiki/Energy\_mix}{energy mix})
   \item understanding and repairing the human body (whole body
     digital twin? neurology?
     \href{https://en.wikipedia.org/wiki/Oncology}{oncology}?
     \href{https://en.wikipedia.org/wiki/Neurosurgery}{neurosurgery}, \href{https://en.wikipedia.org/wiki/Obesity}{obesity})
     \item understanding and improving the Internet (it could be very
       brittle?)
     \item understanding and improving the French law (online, but how much is it consistent?)
     \item improving communication and trust between human beings
   \end{itemize}

 \end{frame}
 
 \begin{frame}
   \frametitle{Why AI systems need to be free software? (1/2)}

   The \href{https://www.gnu.org/philosophy/free-sw.en.htm}{\textsc{Gnu FSF}} defines free software :
   \begin{itemize}

    \item The freedom to run the program as you wish, for any purpose (freedom 0).
    \item The freedom to study how the program works, and change it so it does your computing as you wish (freedom 1). Access to the source code is a precondition for this.
    \item The freedom to redistribute copies so you can help others (freedom 2).
    \item The freedom to distribute copies of your modified versions to others (freedom 3). By doing this you can give the whole community a chance to benefit from your changes. Access to the source code is a precondition for this.

   \end{itemize}

   ``\textcolor{red}{\textbf{What is free software and why it is important to \emph{society}}}''

   \medskip
   
 The ``source code''
 \href{https://en.wikisource.org/wiki/The_Open_Source_Definition}{being
   defined} as : ``The source code must be the preferred form in which
 a programmer would modify the program''
 \end{frame}

 \begin{frame}
   \frametitle{Why AI systems need to be free software? (2/2)}

   See also \href{https://laboutique.edpsciences.fr/produit/1107/9782759824304/Le\%20fabuleux\%20chantier}{
   Le fabuleux chantier : Rendre l’intelligence artificielle
   robustement bénéfique {\relsize{-1.5}{(The fabulous project: making
       artificial intelligence robustly beneficial)}}}

   Notice analogy between free software / open source and academical [mis-]practices

   \bigskip
   

   \begin{itemize}
   \item \href{https://en.wikipedia.org/wiki/Publish_or_perish}{publish or perish}. Pitrat observed that there is not enough insentive to make AI \emph{software systems} and favored experimental approaches.

   \item \href{https://www.nber.org/papers/w7600}{\textit{The Simple Economics of Open Source}}

   \item \href{https://cryptome.org/2015/07/big-other.pdf}{\textit{Big other: surveillance capitalism and the prospects of an information civilization}}
   \end{itemize}

   \medskip
   
   See \href{https://softwareheritage.org/}{\texttt{softwareheritage.org}}
   and dream of mega  knowledge bases
 \end{frame}

 \begin{frame}
   \frametitle{Pitrat's thesis}
   \begin{block}{about intelligence}
     \textbf{There is no \emph{\textcolor{brown}{simple} theory} or \emph{\textcolor{brown}{simple} model} of intelligence} (be it artificial or natural).\\
     $\Rightarrow$ any AI system has to be complex (with chaotic behavior) and evolving!     
 \end{block}

 Pitrat's analogy: a bird can fly, an airplane also fly, they are
 designed differently.


 \begin{itemize}
 \item \textbf{\textcolor{red}{experimentation is king}}

 \item \textbf{\textcolor{red}{theoretical limitations}} (affecting humans too!) \textcolor{red}{do
   not matter \textbf{in practice} :}
   \begin{enumerate}
   \item \href{https://en.wikipedia.org/wiki/Halting_problem}{halting problem}
   \item \href{https://en.wikipedia.org/wiki/Rice\%27s_theorem}{Rice's theorem}
   \item \href{https://en.wikipedia.org/wiki/Gödel's_incompleteness_theorems}{Gödel's incompleteness theorems}

   \item \href{https://en.wikipedia.org/wiki/Church–Turing_thesis}{Church-Turing thesis}

   \item \href{https://en.wikipedia.org/wiki/Curry–Howard_correspondence}{Curry–Howard correspondence}; \\
     see {\href{https://xavierleroy.org/}{\textit{Software, between mind and matter}}} (X.Leroy)

   \item heuristically and \textbf{\textcolor{DarkGreen}{usually}} avoiding \href{https://en.wikipedia.org/wiki/Combinatorial_explosion}{combinatorial explosion}
     
   \end{enumerate}
   
 \end{itemize}
 \end{frame}

 \begin{frame}
   \frametitle{consequences of Pitrat's thesis}

   \begin{relsize}{+1}
 \textcolor{brown}{\textbf{AI systems have to be \emph{globally}
     inconsistent}} (like humans are) and \href{https://en.wikipedia.org/wiki/Antifragile}{\textbf{\emph{antifragile}}} (Taleb)
   \end{relsize}

   \begin{itemize}
     \item importance of ``insight'' :
       \href{https://en.wikipedia.org/wiki/Eureka\_effect}{Eureka
         effect} (``Aha! moment'')
     \item intelligent systems cannot be fully modular or
       compositional to have an
       \href{https://en.wikipedia.org/wiki/Emergence}{emerging}
       intelligent behavior with
       \href{https://en.wikipedia.org/wiki/Self-organization}{self-organization}
       \item theoretical quasi-impossibility to explain or modelize
         intelligence.
       \item AI systems are non-modular and will exhibit
         non-reproducible behavior \\ $\Rightarrow$
         \href{https://en.wikipedia.org/wiki/Randomized_algorithm}{randomized
           algorithms} are practically essential to AI!
       \item \textcolor{brown}{\textbf{time is important in AI systems}} (see 
         \href{http://man7.org/linux/man-pages/man7/time.7.html}{\texttt{time(7)}} on Linux)

         \item importance of
           \href{https://en.wikipedia.org/wiki/Self-awareness}{self-awareness}
           and
           \href{https://en.wikipedia.org/wiki/Introspection}{introspection}
           for intelligent systems (on Linux could be done with
           \href{https://github.com/ianlancetaylor/libbacktrace}{\texttt{libbacktrace}},
           conceptually related to
           \href{https://dl.acm.org/doi/10.1007/BF01019459}{the
             discoveries of continuations})
   \end{itemize}

   Since AI systems need to have some kind of randomness (see
   \href{http://man7.org/linux/man-pages/man4/random.4.html}{\texttt{random(4)}} on Linux),
   \textcolor{red}{\textbf{experimental reproducibility is \emph{ethically} important}}.
 \end{frame}


 \section{Engineering aspects of AI software systems}
 \label{sec:engineering-ai}
 
 \begin{frame}
   \frametitle{Engineering aspects of software}

   We need a computer to run a software:

   \begin{center}
     \includegraphics[width=0.3\textwidth]{black-box-computer-small}\\
     (photo by \href{http://matthieu-starynkevitch.com/}{Matthieu Starynkevitch})
   \end{center}

   But J.Pitrat wrote (in \href{https://onlinelibrary.wiley.com/doi/book/10.1002/9780470611791}{\textit{Artificial Beings}}, §4 p67):
   \begin{quote}
     The black box is the opposite of consciousness, and we must \textcolor{brown}{always} avoid it
   \end{quote}
   and later (\textit{Artificial Beings} p259):
     \begin{quote}
       \textsc{Caia} uses \textcolor{brown}{two} programs that it has not written: \href{http://gcc.gnu.org/}{\textsc{Gcc}} and \href{https://https://en.wikipedia.org/wiki/Linux}{\textsc{Linux}}
     \end{quote}
     
   {\relsize{-2}{emphasis is mine; I will dare doing some nitpicking}}

   My black box runs \textsc{Caia} {\large \smiley{}}
 \end{frame}


 
  \end{document}

%%%%%%%%%%%%%%%%%%%%%%%%%%%%%%%%%%%%%%%%%%%%%%%%%%%%%%%%%%%%%%%%
%% Local Variables: ;;
%% compile-command: "./build.sh" ;;
%% End: ;;
%%%%%%%%%%%%%%%%%%%%%%%%%%%%%%%%%%%%%%%%%%%%%%%%%%%%%%%%%%%%%%%%
